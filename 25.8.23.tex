\chapter{25.8.23}
\section{财经资讯}
\vspace{1em}
\noindent\textbf{阅读全文:} \url{https://mp.weixin.qq.com/s/HsvNrUwIcCJJl2V10TIzdQ}
\subsection{政策动态}
\begin{itemize}
    \item \textbf{稀土管理:}工信部等三部门发布《稀土开采和冶炼分离总量调控管理暂行办法》,对稀土开采及冶炼分离实行总量调控。
    \item \textbf{光伏行业:}中国光伏行业协会呼吁抵制低价恶性竞争,要求提高技术评标权重,推动高质量竞争。
    \item \textbf{"三北"工程:}国务院通过《"三北"工程总体规划》,强调筑牢北方生态安全屏障,配套财税土地支持政策。
    \item \textbf{体育消费:}国常会研究释放体育消费潜力,鼓励举办消费活动、实施惠民举措。
\end{itemize}

\subsection{宏观经济}
\begin{itemize}
    \item \textbf{基建进展:}华南最大盾构产业基地(中国铁建)在清远建成投产。
    \item \textbf{航天计划:}俄罗斯宣布2028年实施"月球-26"探测任务,推迟金星探测至2036年。
    \item \textbf{贸易关税:}欧盟葡萄酒烈酒未纳入美欧降税清单,15\%关税维持,欧盟预估损失超20亿欧元。
    \item \textbf{美联储政策:}鲍威尔称就业下行风险上升,政策调整可能性增大。
\end{itemize}

\subsection{地产动态}
\begin{itemize}
    \item \textbf{城市更新:}中央部署城市更新向治理范式变革转型,释放经济潜力。
    \item \textbf{恒大进展:}广州中院受理恒大地产广东公司破产清算,此前113亿债权被挂牌。
    \item \textbf{政策支持:}金融监管总局支持福建银行为台胞提供住房按揭贷款;沈阳优化灵活就业者公积金政策。
\end{itemize}

\subsection{股市盘点}
\begin{itemize}
    \item \textbf{指数表现:}A股三大指数集体上涨(沪指+1.45\%,创业板+3.36\%),成交额2.55万亿;恒指涨0.93\%。
    \item \textbf{资金动向:}
    \begin{itemize}
        \item 恒生指数调入中国电信、京东物流、泡泡玛特
        \item 中信证券:预计4.5-9万亿存款或转向"固收+"产品间接入市
        \item 两市融资余额减少9.71亿元
    \end{itemize}
    \item \textbf{公司业绩:}
    
\begin{itemize}
        \item 大幅增长:中国中车(净利+72.48\%)、洛阳钼业(+60.07\%)、神工股份(+925.55\%)
        \item 显著下滑:万科A(亏损119亿)、通威股份(亏损49.55亿)、酒鬼酒(净利-92.6\%)
    \end{itemize}
    \item \textbf{产业布局:}景旺电子拟50亿投建珠海扩产项目;汇绿生态子公司签450万支光模块基地合同。
\end{itemize}

\subsection{行业观察}
\begin{itemize}
    \item \textbf{新能源:}全国充电桩达1669.6万个(+53\%),私桩增速(+58.8\%)快于公桩(+38\%)。
    \item \textbf{科技产业:}
    \begin{itemize}
        \item 液冷技术渗透率预计从2024年14\%升至2025年33\%(TrendForce)
        \item 全球网络安全市场2029年将达4162亿美元(CAGR 11.2\%)
    \end{itemize}
    \item \textbf{医药创新:}我国在研创新药占全球30\%,"十四五"批准387个儿童药、147个罕见病药。
    \item \textbf{农业:}早稻亩产首破400公斤创历史新高。
\end{itemize}

\subsection{公司要闻}
\begin{itemize}
    \item \textbf{蔚来:}李斌称降价是为生存,新款ES8成本降低仍保持毛利。
    \item \textbf{特斯拉:}接入字节火山引擎大模型,Model Y L搭载豆包与DeepSeek模型。
    \item \textbf{渠道拓展:}喜茶全国4000家门店上线饿了么;五粮液推"一见倾心"新品瞄准年轻群体。
    \item \textbf{阿里调整:}业务重组为四大板块,饿了么、飞猪归入阿里中国电商集团。
\end{itemize}

\subsection{环球视野}
\begin{itemize}
    \item \textbf{极端气候:}欧盟年内过火面积达101万公顷创新高,西班牙占四成。
    \item \textbf{产业政策:}韩国2026年拟投251亿美元研发;美国防部计划5亿美元储备战略钴。
    \item \textbf{公共卫生:}孟加拉国登革热死亡病例达110例;美国疾控中心裁员约600人。
\end{itemize}

\subsection{金融数据}
\begin{itemize}
    \item \textbf{汇率:}在岸人民币报7.1805,中间价调贬至7.1321。
    \item \textbf{商品:}燃料油、烧碱涨超2\%,碳酸锂跌超4\%。
    \item \textbf{债市:}10年期国债收益率涨1.51BP至1.78\%。
    \item \textbf{全球市场:}
    \begin{itemize}
        \item 美股集体收涨(道指+1.89\%,纳指+1.88\%)
        \item 原油小幅上涨(WTI报63.66美元/桶)
    \end{itemize}
\end{itemize}

\subsection{财税理论探讨}
\subsubsection{美国国债偿还机制}
\textbf{核心观点:} "内债不是债,是税"——基于宏观税负>0的前提,货币流通N次后大部分终将通过税收回流政府。
\begin{enumerate}
    \item \textbf{货币循环逻辑:} 政府发行国债获取资金 → 资金进入经济循环 → 征税回收货币(部分)
    \item \textbf{实质偿还方式:}
    \begin{itemize}
        \item 债务展期:发行新债偿还旧债(依赖主权信用)
        \item 通胀稀释:通过货币政策降低实际债务价值
        \item 税收回流:长期税基增长保障偿付能力
    \end{itemize}
    \item \textbf{关键前提:} 需维持经济增速 > 债务利率,否则陷入债务螺旋。
\end{enumerate}


\section{蔚来战略转向:降价求生}
\subsection{核心表态与战略调整}
\begin{itemize}
    \item \textbf{生存优先:}李斌坦言"如果蔚来继续保持高价,很难参与市场竞争",新款ES8成本降低后仍有毛利
    \item \textbf{用户取舍:}老用户劝其"不要太在乎感受",公司生存是对80万用户的最大责任
    \item \textbf{业绩目标:}2024年最佳目标是盈利,退而求其次为减少亏损
\end{itemize}

\subsection{市场拐点预判}
\begin{itemize}
    \item \textbf{纯电拐点:}2025年将成为纯电SUV体验收益>充电不便损失的转折点
    \item \textbf{竞争格局:}纯电大三排SUV增速已超增程/插混/燃油车型(如问界M9月销超15万辆,理想L系列月均2.5万辆)
    \item \textbf{终极方案:}李斌断言"可换电的纯电大三排SUV是终极选择"
    \item \textbf{订单表现:}秦力洪透露全新ES8小时级订单量超乐道L90
\end{itemize}

\subsection{价格调整详情}
\begin{center}
\begin{tabular}{|l|c|c|c|}
\hline
\textbf{车型方案} & \textbf{全新ES8} & \textbf{老款ES8} & \textbf{降幅} \\
\hline
整车购买 & 41.68-45.68万 & 49.8-57.8万 & ↓8.12万起 \\
\hline
BaaS方案 & 30.88-34.88万 & 42.8-47万 & ↓11.92万起 \\
\hline
\textbf{电池配套} & \textbf{新价格} & \textbf{原价格} & \\
\hline
100度电池包 & 10.8万 & 12.8万 & ↓2万 \\
\hline
永久升级服务 & 3.8万 & 5.8万 & ↓2万 \\
\hline
\end{tabular}
\end{center}

\subsection{战略逆转背景}
\begin{enumerate}
    \item \textbf{历史立场:}2021年李斌明确"不降价"原则,认为降价是特斯拉的策略
    \item \textbf{市场挤压:}
    \begin{itemize}
        \item 竞品策略:问界/理想采用"高配+中定价"抢占市场
        \item 豪华围剿:BBA电动车型降价直接挤压蔚来价格区间
    \end{itemize}
    \item \textbf{销量困境:}7月主品牌仅交付1.27万辆(总量2.1万含子品牌)
    \item \textbf{财务危机:}
    
\begin{itemize}
        \item 累计亏损突破1000亿元
        \item 2025年Q1净亏损68.91亿(同比扩大30.19\%)
    \end{itemize}
\end{enumerate}

\subsection{战略启示}
    \begin{enumerate}
        \item \textbf{价格循环:}降价→扩大用户基数→通过服务(BaaS/升级)实现税收式回流
        \item \textbf{生存公式:}市场规模增速 > 亏损扩大速度
        \item \textbf{终极目标:}建立"用户生态税基"(换电网络/服务订阅)
    \end{enumerate}

    
