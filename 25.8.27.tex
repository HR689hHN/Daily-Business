\chapter{25.8.27}
\vspace{1cm}
\noindent\textbf{阅读全文:(微信文章)} \url{https://mp.weixin.qq.com/s/q5t1hQ7GHPz5tpxpC01XZQ}

\section{2025年8月27日《财经早餐》全景速读}
\textbf{一句话总结:}  
国务院再推“人工智能+”顶层设计,ETF规模首破5万亿元;暑期消费品质升级、地产政策托底、A股缩量震荡、海外金价续创新高,成为当日宏观与市场的四大主线。

\subsection{政策与宏观}
\begin{enumerate}[leftmargin=*, nosep]
    \item \textbf{国务院}\\
    发布《深入实施“人工智能+”行动的意见》:  
    2027年率先实现AI与6大重点行业深度融合,智能终端/智能体普及率超70\%;  
    2030年普及率超90\%,并配套财政、金融、长期资本支持。
    \item \textbf{总理李强:}\\  
    主动扩大优质服务进口,推动服务贸易制度型开放,优化跨境资金管理与数据流动。
    \item \textbf{发改委}答记者问:  \\
    建设国家人工智能应用中试基地,降低应用创新门槛;对外提出“人工智能+”全球合作新模式。
    \item \textbf{商务部}:  \\
    截至2024年底,中国对外直接投资存量超3万亿美元,连续8年全球前三。
    \item \textbf{能源局}:  \\
    7月单月用电量首破1万亿千瓦时,相当于日本全年用电量,迎峰度夏总体平稳。
\end{enumerate}

\subsection{资金面与资本市场}
\begin{enumerate}[leftmargin=*, nosep]
    \item \textbf{ETF里程碑}  \\
    中国ETF规模正式突破5万亿元(5.07万亿元),仅用4个月完成4万亿→5万亿跨越;  
    全市场1271只ETF中,101只规模超百亿、6只超千亿。
    \item \textbf{A股表现}  \\
    周二缩量至2.68万亿元(-4621亿元),个股涨多跌少;  
    上证-0.39\%报3868,深证+0.26\%报12473,创业板指-0.75\%报2742;  
    领涨:猪肉、游戏、消费电子、美容护理;领跌:CRO、稀土永磁、军工。
    \item \textbf{港股}  \\
    恒指-1.18\%,恒生科技-0.74\%;南向资金净买入165.73亿港元。
    \item \textbf{两融余额}  \\
    截至8月25日,两市融资余额21655.94亿元,日增328亿元续创新高。
    \item \textbf{公募基金}  \\
    7月底总规模35.08万亿元,年内第十次创历史新高;货基、股基、混基均显著增长,债基略降。
\end{enumerate}

\subsection{产业与公司}
\begin{enumerate}[leftmargin=*, nosep]
    \item \textbf{AI算力与芯片}  \\
    寒武纪上半年营收28.81亿元,同比+4348\%,净利润10.38亿元扭亏;  
    阿里云百炼下调大模型缓存价格,输入Token缓存费降至原价20\%。
    \item \textbf{消费电子/汽车}  \\
    IDC:2025年全球折叠屏手机出货1983万台,2029年或达2729万台,CAGR≈7.8\%;  
    小鹏汽车何小鹏:国内汽车行业淘汰赛还剩五年,最终或仅剩五家中国车企。
    \item \textbf{影视与文旅}  \\
    暑期档票房突破112亿元、观影人次破3亿,均创纪录;  
    飞猪:暑期订单均价同比+9.9\%,学生和家庭客群驱动品质游。
    \item \textbf{大宗与资源}  \\
    钨价延续涨势,65\%黑钨精矿23.3万元/标吨,年初至今+62.9\%;  
    紫金矿业上半年净利232.9亿元,同比+54.4\%;中国石油中期派息402.65亿元。
    \item \textbf{并购重组}  \\
    必易微拟2.95亿元收购上海兴感半导体100\%股权;  
    南新制药拟现金收购未来医药资产组,构成重大资产重组。
    \item \textbf{风险警示}  \\
    新华锦提示可能被实施其他风险及退市风险警示;  
    浙文影业独立董事刘静被留置,公司称事项与己无关。
\end{enumerate}

\subsection{地产与基建}
\begin{enumerate}[leftmargin=*, nosep]
    \item \textbf{政策监管}  \\
    住建部、央行联合发文,房地产从业机构须履行反洗钱义务,不得向身份不明客户售房或提供经纪服务。
    \item \textbf{老旧小区改造}  \\
    1-7月全国新开工改造1.98万个小区,河北、辽宁、重庆、安徽、江苏、上海开工率超90\%。
    \item \textbf{土地与购房激励}  \\
    北京顺义区宅地底价10.3亿元成交,楼面价27989元/㎡;  
    安徽淮南出台购房补贴、人才购房、团购支持等促高质量发展措施。
\end{enumerate}

\subsection{商品与全球宏观}
\begin{enumerate}[leftmargin=*, nosep]
    \item \textbf{能源}  \\
    国内成品油新一轮下调:92号汽油每升降0.14元,50L油箱加满省7元;  
    国际油价大跌:WTI-2.39\%报63.25美元,布伦特-2.29\%报67.22美元。
    \item \textbf{贵金属}  \\
    COMEX黄金期货+0.65\%报3439.6美元/盎司,续创新高。
    \item \textbf{汇率}  \\
    在岸人民币兑美元收报7.1621,日跌0.1454\%;中间价7.1188。
    \item \textbf{海外股市}  \\
    美股三大指数小幅收涨:道指+0.30\%,纳指+0.44\%,标普+0.41\%;  
    欧股多数收跌,法国CAC40-1.70\%;亚太日经-0.97\%,韩国-0.95\%。
\end{enumerate}

\clearpage

\section{深度洞察:2025年8月27日《财经早餐》内核拆解}
\textbf{一句话总结:}  \\
“AI+财政”双轮驱动的政策范式确立,居民财富迁徙从地产—理财—ETF完成闭环,产业侧进入“剩者为王”淘汰赛,全球滞胀阴影下黄金成为终极押注。

\subsection{政策范式切换:从“土地财政”到“算力财政”}
\begin{enumerate}[leftmargin=*, nosep]
    \item \textbf{财政支出结构}  \\
    国务院《“人工智能+”行动意见》首次把AI算力、芯片、智能终端列入中央—地方共同事权,意味着土地出让金下滑后,政府将用“算力券”“模型券”替代传统“土地返税”。  
    \item \textbf{金融资源再分配}  \\
    “长期资本、耐心资本、战略资本”三词并列,暗示政策性银行、养老基金、主权财富基金将成为AI基础设施的“类土地”估值锚,未来算力中心 REITs 有望复制2020–2022年保障性租赁住房 REITs 的估值溢价路径。
\end{enumerate}

\subsection{居民资产负债表:一场静默的“财富搬家”}
\begin{enumerate}[leftmargin=*, nosep]
    \item \textbf{ETF突破5万亿}  \\
    仅用4个月完成1万亿增量,核心来源并非新发基金,而是“老基民赎回主动管理→买入指数”与“理财净值化后资金溢出”共振,标志着中国个人投资者第一次大规模放弃“地产信仰+刚兑信仰”双锚。  
    \item \textbf{两融余额2.17万亿}  \\
    杠杆资金与ETF同步新高,反映居民开始用“券商两融+ETF”替代“首付+按揭”的杠杆结构,风险权重由不动产转向流动性资产。
\end{enumerate}

\subsection{产业生死线:折叠屏、机器人、汽车进入“清场时刻”}
\begin{enumerate}[leftmargin=*, nosep]
    \item \textbf{折叠屏}  \\
    IDC预测五年CAGR仅7.8\%,远低于2021–2024年的30\%+,说明行业渗透率天花板提前出现,手机品牌押注MR与AI眼镜实为“第二增长曲线焦虑”。
    \item \textbf{人形机器人芯片}  \\
    2028年市场规模4800万美元,绝对值仍小,但NVIDIA Jetson Thor算力达2070 TFLOPS为上一代7.5倍,意味着“硬件预埋+软件订阅”将成为机器人行业唯一存活模式——硬件不赚钱,数据与算法迭代权决定生死。
    \item \textbf{汽车行业}  \\
    何小鹏“五年剩五家”言论与比亚迪、吉利、小米的港股回购形成互文:淘汰赛已进入“现金流+供应链”双重挤压阶段,二线车企将在2026年前完成破产重组或被收购。
\end{enumerate}

\subsection{全球滞胀交易:金油比发出衰退预警}
\begin{enumerate}[leftmargin=*, nosep]
    \item \textbf{金油比}  \\
    黄金3439美元 vs WTI 63美元,金油比升至54.4,为1974年以来前5\%分位,历史经验显示6–12个月内美国经济衰退概率>60\%。  
    \item \textbf{中国电力}  \\
    单月用电1万亿千瓦时≈日本全年用电量,侧面验证“世界工厂”产能利用率仍在高位;若外需回落,高基数下电力增速或快速转负,成为观测全球衰退向中国传导的先行指标。
\end{enumerate}

\subsection{人民币与资本账户:贬值压力中的“结构性保护”}
\begin{enumerate}[leftmargin=*, nosep]
    \item \textbf{中间价 vs 收盘价背离}  \\
    中间价7.1188,收盘价7.1621,日内偏离-0.6\%,为2024年Q3以来最大,央行通过“汇率缓冲区”释放外需放缓压力,避免一次性贬值冲击居民财富搬家进程。  
    \item \textbf{南向资金}  \\
    净买入165亿港元却难挡恒指下跌,反映外资在“金油比预警+中概退市阴影”下持续减配港股,南下资金成为“边际定价者”,短期波动将与A股ETF资金形成跷跷板。
\end{enumerate}

\clearpage

\section{江苏国泰“120亿理财+18亿炒股”事件全景速读}
\vspace{1cm}
\noindent\textbf{阅读全文:(微信文章)} \url{https://mp.weixin.qq.com/s/NBKF5Pzp4g8lW26eQR3gJQ}

\textbf{一句话总结:}  
一家主业清晰的出口型实业龙头,在账面125亿现金的“甜蜜负担”下,先拟巨资炒股、旋即叫停并承诺高分红,折射出A股上市公司“资金配置焦虑”与监管导向的即时博弈。

\subsection{事件时间轴}
\begin{enumerate}[leftmargin=*, nosep]
    \item \textbf{8月24日}\\
    公司公告  
    拟用不超过120亿元闲置自有资金进行委托理财(中低风险,单笔最长36个月);  
    拟用不超过18.3亿元进行证券投资,其中紫金科技拟以15亿元设立张家港鼎瑞投资专攻股票。
    \item \textbf{8月26日}\\
    舆情发酵  
    市场质疑“主业出口+化工”与“巨资炒股”割裂;叠加公司同日宣布终止40万吨电解液项目,加深“不务正业”印象。
    \item \textbf{8月26日晚} \\
    公司紧急转向  终止15亿元设立证券投资子公司;同步抛出2025–2027年股东回报规划:每年现金分红不少于可分配利润40\%(此前仅10\%)。
\end{enumerate}

\subsection{资金画像}
\begin{enumerate}[leftmargin=*, nosep]
    \item \textbf{现金体量}  \\
    2025年中报货币资金125.7亿元,占市值约90\%,占营收约68\%,现金流极度充裕。
    \item \textbf{理财收益}  \\
    上半年依托结构性存款等低风险投资实现收益约1.2亿元,占利润总额9.75\%,已成为盈利“第二曲线”。
    \item \textbf{证券投资成绩单}  \\
    计入权益的累计公允价值变动亏损7195.6万元,坐实“炒股输多赢少”。
\end{enumerate}

\subsection{主业与战略矛盾}
\begin{enumerate}[leftmargin=*, nosep]
    \item \textbf{业务结构}  \\
    供应链服务(纺织品服装)占比约95\%,其中83\%收入来自海外;化工新能源占比仍小。  
    证券投资与主业缺乏协同,属跨界博弈。
    \item \textbf{产能项目终止}  \\
    40万吨电解液项目因“外部客观条件及行业环境变化”搁浅,显示新能源赛道拥挤、盈利预期下调,公司选择“现金为王”。
\end{enumerate}

\subsection{监管与市场反馈}
\begin{enumerate}[leftmargin=*, nosep]
    \item \textbf{监管导向}  \\
    监管层持续强调“聚焦主业、提升投资者回报”,江苏国泰快速转向高分红,可视作对政策信号的即时响应。
    \item \textbf{投资者情绪}  \\
    公告后首个交易日公司股价未出现明显波动,反映市场对其“认错”动作基本认可,短期舆情风险降温。
\end{enumerate}

\subsection{启示与结论}
\begin{enumerate}[leftmargin=*, nosep]
    \item 对上市公司:  \\
    在主业再投资回报率下降、理财收益率下行背景下,如何平衡现金效率与监管红线,将成为持续难题。
    \item 对投资者:  \\
    高现金+低分红的历史模式正在改变,需关注企业资本配置逻辑突变带来的估值重估机会与风险。
\end{enumerate}
